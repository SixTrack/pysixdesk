\begin{titlepage}
\begin{center}\normalsize\scshape
    European Organization for Nuclear Research \\
    CERN BE/ABP
\end{center}
\vspace*{2mm}
\begin{flushright}
    CERN/**/** \\
    Updated October 2019
\end{flushright}
\begin{center}\Huge
    \textbf{PySixDesk} \\
    \LARGE Version 1.0 \\
    \vspace*{8mm}The running Environment for SixTrack \\
    \vspace*{8mm}\textbf{User's Reference Manual}
\end{center}
\begin{center}
    X.~Lu (CSNS, CERN),
    A.~Mereghetti (CERN),
    L.~Coyle (EPFL, CERN)
\end{center}
\begin{center}\large
    \vspace*{10mm}\textbf{Abstract} \\
\end{center}
The aim of PySixDesk is to manage and control massive sixtrack simulations starting from a MADX input file. 

\vfill
\begin{center}
    Geneva, Switzerland \\
    \today
\end{center}

\end{titlepage}
